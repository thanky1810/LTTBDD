\documentclass{article}

% Gõ tiếng Việt với XeLaTeX
\usepackage{fontspec}
\setmainfont{Times New Roman} % hoặc Times New Roman/Arial

% Khổ giấy & lề
\usepackage[a4paper,top=2cm,bottom=2cm,left=3cm,right=3cm,marginparwidth=1.75cm]{geometry}

% Các gói hay dùng
\usepackage{amsmath}
\usepackage{graphicx}
\usepackage[colorlinks=true, allcolors=blue]{hyperref}
\usepackage{microtype} % làm văn bản đẹp hơn (tùy chọn)

\title{Bài tập tuần 1}
\author{Thân Văn Ký - 089205011695}

\begin{document}
\maketitle

\section{Mong muốn và định hướng của bạn là gì sau khi học xong môn này?}
App không phải là định hướng của em, em theo chuyên ngành blockchain và web3, mong muốn sau môn này có thể hiểu hơn về app, những kiến thức làm việc thực tế từ thầy.

\section{Theo bạn, trong tương lai gần (10 năm) lập trình di động có phát triển không? Giải thích tại sao?}
Theo ý kiến của em trong 10 năm tới lập trình di động sẽ cực kì phát triển vì sẽ có nhiều công cụ hỗ trợ như là AI, cùng với đó là việc lập trình được hỗ trợ tối đa với nhiều công cụ hữu ích, và có nhiều ví dụ minh họa từ những người lập trình đi trước. Việc nhiều ứng dụng đã được xây dựng thì mức độ cạnh tranh này càng cao kéo theo sự phát triển của ngành. Nhưng song song với đó là lập trình viên di động ngày càng phải có tay nghề cao hoặc ý tưởng đột phá, vì việc lập trình càng dễ thì nhiều người có thể làm được nên cạnh tranh rất cao.

\end{document}
